\section{Conclusion and Future Work}
\label{sec:conclusion}

%\feedback{@Oscar: this conclusion section is missing a statement (paragrph) of what can be derived from the experiments run, on what are the envisaged challenges that %you foresee, etc.}

%\feedback{@Oscar: Something that should be discussed in a discussion or future
%work section is how we can build a library of descriptions for all the workflows
%in the world (is it possible? how many descriptions can be reused? how much reuse?}

In this work, we proposed a semantic modeling approach to conserve computational environments in scientific workflow executions, where  the resources involved in the execution of the experiment are described using a set of semantic vocabularies. We defined and implemented 4 domain ontologies, aggregated in the the WICUS ontology network. From these models, we defined a process for documenting workflow applications, the workflow management system where they can be executed, and their dependencies. 

We conducted experiments with three workflow applications from different areas of science (Montage, Epigenomics and SoyKB) using the Pegasus WMS. We executed the ISA to obtain a set of PRECIP and Vagrant scripts to describe and execute the experiment. Experimental results show that our approach can reproduce an equivalent execution environment of a predefined VM image on academic, public, and local Cloud platforms.

Semantic annotations of the computational environment, combined with the ISA and the scripting functionality provided by PRECIP and Vagrant, is a powerful approach for achieving reproducibility of computational environments in future experiments, and at the same time addresses the challenges of high storage demand of VM images. \rev{In this work we have shown how by using a generic virtual machine image, only containing the operating system and its default associated tools, we have been able to dynamically generate three different virtual machines, each of them fulfilling the requirements of a workflow. These images are public and were already available and supported by the different providers we have worked with. Thus, we have saved several gigabytes by not generating an image for each workflo.}
\note{Not sure how to quantify the amount of GB we save. We could ask Mats for the size of the WM Images at FG, but not sure how to calculate the size of the AWS and Vagrant VMs}

\rev{This kind of solution requires the workflow management system and the scientific applications related to the workflow to be publicly available. We argue that this necessary condition holds in most cases, being the majority of software solutions developed in science open and public}. \rev{For achieving the reproducibility of the workflow execution environment, as we have done in this work, access to the resources describing the workflow is required. The information about the software components involved on the execution must be available and as explicit as possible. In the case of the experiments of this work, the Pegasus Transformation Catalog file provided this information, which is also available in other workflow systems using similar files.}

\rev{Also, knowledge about semantic technologies is required for generating some of the annotations for describing the underlaying infrastructure. At the time of this writing the description of the WMS and the computational resources are generated manually, whereas the rest of the descriptions are annotated by the tools of the WICUS system. Future versions of the system might include tools for fully automating this process. Nevertheless, as those annotations are reused, we argue that this effort is reasonable. Moreover, this process may be assisted by experts on these technologies, in the same way librarians help book authors and publishers to conserve their contributions.}


%The drawback of our approach is that it assumes that the application and the workflow management system binaries are publicly available.

The results of this work also show how components, such as the workflow system, can be annotated once and then reused among workflows. We envision a library of workflow descriptions in which components and tools can be easily reused, even during the development process of the workflow. Many workflows are built upon previous workflows, especially within the context of a scientific domain, and hence having such type of library would be helpful. We plan to study how and when those libraries can be built, analyzing their degree of reuse. \rev{As introduced in this work, our descriptions of the environment, generated using the WICUS ontology network, is compliant with the Linked Data principles. This would allow its future publication and integration with other information sources, which fits the idea of a distributed and structured library of descriptions that could be publicly available worldwide.}

\rev{The reproducibility of a computational environment presupposes that at least some differences may appear when compared to the former counterpart. In our work, as we are using virtualization technologies and dynamic deployment of software components, we assume that these differences will appear. We are obtaining an equivalent infrastructure, which might not be an exact copy of the original one, but that expose the required characteristics for executing the workflow. These kind of approach would allow scientists to understand how the underlaying infrastructure is defined and even to introduce changes over it to test their effect on the experiment, as the configuration parameters of the environment are explicitly defined in the semantic annotations.}

\rev{We are currently studying the applicability of our approach to other workflows, belonging to other scientific areas, and applying it to new workflow management systems. Even thought the cloud systems included in this work cover a representative spectrum of cloud providers, we are studying its applicability to other cloud scenarios and emerging container-based solutions. Also, we are planning to extend the scope of our approach to cover more complex and large execution infrastructures. From multi-node clusters to infrastructures requiring specific network topologies, we will study how our approach can reproduce those characteristics.}


We will also work on increasing  the degree of automation of the semantic annotation process to describe both the workflow application and the workflow management system. As introduced in this work, WICUS is an ongoing effort, thus we also plan to extend the ontology network to include new concepts and relations such as software variants, incompatibilities, and user policies for resource consumption.




