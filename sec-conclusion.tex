\section{Conclusion and Future Work}
\label{sec:conclusion}

%\feedback{@Oscar: this conclusion section is missing a statement (paragrph) of what can be derived from the experiments run, on what are the envisaged challenges that you foresee, etc.}

%\feedback{@Oscar: Something that should be discussed in a discussion or future
work section is how we can build a library of descriptions for all the workflows
in the world (is it possible? how many descriptions can be reused? how much reuse?}

In this work, we proposed a semantic modeling approach to conserve computational environments in scientific workflow executions, where  the resources involved in the execution of the experiment are described using a set of semantic vocabularies. We defined and implemented 4 domain ontologies, aggregated in the the WICUS ontology network. From these models, we defined a process for documenting workflow applications, a workflow management system, and their dependencies. 

We conducted and experimental process in which we studied three workflow applications from different areas of science (Montage, Epigenomics and SoyKB) using the Pegasus WMS. We executed the ISA to obtain a set of PRECIP and Vagrant scripts to describe and execute the experiment. Experimental results show that our approach can reproduce an equivalent execution environment of a predefined VM image on academic, public, and local Cloud platforms.

Semantic annotations of the computational environment, combined with the ISA and the scripting functionality provided by PRECIP and Vagrant, is a powerful approach for achieving reproducibility of computational environments in future experiments, and at the same time addresses the challenges of high storage demand of VM images. The drawback of our approach is that it assumes the application and the workflow management system binaries are publicly available.

The results of this work also show how components, such as the workflow system, can be annotated once and then reused among workflows. We envision a library of workflow descriptions in which components and tools can be easily reused, even during the development process of the workflow. Many workflows are built upon previous workflows, specially within the context of a scientific domain, and hence having such kind of library would be helpful. We plan to study how and when those libraries can be built, analyzing their degree of reuse.

We also aim to study different workflows, belonging to different scientific areas, as well as applying our approach to new workflow management systems. We will also work on increasing  the degree of automatization of the semantic annotations process to describe both the workflow application and the workflow management system. As introduced in this work, WICUS is a ongoing effort, thus we also plan to extend the ontology network to include new concepts and relations such as software variants, incompatibilities, and user policies for resource consumption.