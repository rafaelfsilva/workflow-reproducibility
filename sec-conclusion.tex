\section{Conclusion and Future Work}
\label{sec:conclusion}

\feedback{@Oscar: this conclusion section is missing a statement (paragrph) of what can be described from the experiments run, on what are the envisaged challenges that you foresee, etc.}

In this work, we proposed a semantic modeling approach to conserve computational environments in scientific workflow executions, where  the resources involved in the execution of the experiment are described using a set of semantic vocabularies. We defined and implemented 4 domain ontologies, aggregated in the the WICUS ontology network. From these models, we defined a process for documenting workflow applications, a workflow management system, and their dependencies. We conducted and experimental process in which we studied three workflow applications from different areas of science (Montage, Epigenomics and SoyKB) using the Pegasus WMS. We executed the ISA to obtain a set of PRECIP and Vagrant scripts to describe and execute the experiment. Experimental results show that our approach can reproduce an equivalent execution environment of a predefined VM image on academic, public, and local Cloud platforms.

The semantic annotations of the computational environment, combined with the ISA and the scripting functionality provided by PRECIP and Vagrant, is a powerful approach for achieving reproducibility of computational environments in future experiments, and at the same time addresses the challenges of high storage demand of VM images. The drawback of our approach is that it assumes the application and the workflow management system binaries are publicly available.

In the future we plan to study different workflows, belonging to different scientific areas, as well as applying our approach to new workflow management systems. We also aim to increase the degree of automatization of the semantic annotations process to describe both the workflow application and the workflow management system. As introduced in this work, WICUS is a ongoing effort, thus we also plan to extend the ontology network to include new concepts and relations such as software variants, incompatibilities, and user policies for resource consumption.