\section{Conclusion and Future Work}
\label{sec:conclusion}

In this work, we proposed a semantic modeling approach to conserve 
computational environments in scientific workflow executions, where  
the resources involved in the execution of the experiment are described 
using a set of semantic vocabularies. We defined and implemented four 
domain ontologies, aggregated in the the WICUS ontology network. From 
these models, we defined a process for documenting workflow applications, 
the workflow management system, and their dependencies.

We conducted experiments with three real workflow applications (Montage, 
Epigenomics, and SoyKB) from different sciences using the Pegasus WMS. 
We used the Infrastructure Specification Algorithm (ISA) to obtain a set of 
PRECIP and Vagrant scripts to describe and execute the experiment. 
Experimental results show that our approach can reproduce an equivalent 
execution environment of a predefined VM image on academic, public, and 
local Cloud platforms.

Semantic annotations of the computational environment, combined with the 
ISA and the scripting functionality provided by PRECIP and Vagrant, is a 
powerful approach for achieving reproducibility of computational environments 
in future experiments, and at the same time addresses the challenges of high 
storage demand of VM images. In this work, we have demonstrated that
an equivalent computational environment (that fulfills the requirements of a
workflow) can be obtained from simple base VM images, which are already 
available in most Cloud platforms. Consequently, there is no need to create 
and store a new VM image for each workflow. Therefore, this approach 
significantly diminishes the costs and efforts in maintaining a VM image that 
may have limited usage. We acknowledge that this approach may be limited 
to the online (publicly) availability of the software, however most of the scientific
tools developed nowadays are hosted as an open-source project.

% how by using a generic VM image, only containing the operating system and its default associated tools, we have been able to dynamically generate three different virtual machines, each of them fulfilling the requirements of a workflow. These images are public and were already available and supported by the different providers we have worked with. Thus, we have saved several gigabytes by not generating an image for each workflow.}
%\note{Not sure how to quantify the amount of GB we save. We could ask Mats for the size of the WM Images at FG, but not sure how to calculate the size of the AWS and Vagrant VMs}

%\rev{This kind of solution requires the workflow management system and the scientific applications related to the workflow to be publicly available. We argue that this necessary condition holds in most cases, being the majority of software solutions developed in science open and public}. 

To attain reproducibility, the proposed model requires that information about
the software components involved on the workflow execution should be available
and as explicit as possible. In Pegasus, the transformation catalog provides 
all necessary information. Similar catalogs (or files) are also available in other
workflow systems. 
%\rev{For achieving the reproducibility of the workflow execution environment, as we have done in this work, access to the resources describing the workflow is required. The information about the software components involved on the execution must be available and as explicit as possible. In the case of the experiments of this work, the Pegasus Transformation Catalog file provided this information, which is also available in other workflow systems using similar files.}
Also, knowledge about semantic technologies is required for generating 
some of the annotations for describing the underlying infrastructure. Currently,
the description of the WMS and the computational resources are manually 
generated, whereas the remaining annotations are performed by tools of the 
WICUS system. Future work include the development of tools to fully (or 
partially) automate this process. The annotation process may be assisted 
by experts, in the same way librarians aid book authors and publishers to 
conserve their contributions.
%Nevertheless, as those annotations are reused, we argue that this effort is reasonable.

The results of this work also show how components, such as the workflow 
system, can be annotated once and then reused among workflows. We 
envision a library of workflow descriptions in which components and tools 
can be easily reused, even during the development process of the workflow. 
Many workflows are built upon previous workflows, especially within the 
context of a scientific domain, and hence having such type of library would 
be helpful. We plan to study how and when those libraries can be built, 
analyzing their degree of reuse. As introduced in this work, our 
descriptions of the environment, generated using the WICUS ontology 
network, is compliant with the Linked Data principles. This would allow its 
future publication and integration with other information sources, which fit 
the idea of a distributed and structured library of descriptions that could be 
publicly available worldwide.

We are currently studying the applicability of our approach to other 
workflow applications (from different scientific areas) and systems. Although
the Cloud systems included in this work cover a representative spectrum of 
Cloud providers, we are also considering other Cloud scenarios and 
emerging container-based solutions. Also, we plan to extend the scope of 
our approach to enable reproducibility of more complex and larger execution 
infrastructures, such as multi-node clusters and topology-specific network 
platforms.
% New cloud management systems and enactment files will be also explored in the future, aiming to evaluate their performance and suitability according to the goals of our work.}
% We will study how our approach can reproduce those characteristics.

We will also work on increasing  the degree of automation of the semantic annotation process to describe both the workflow application and the workflow management system. As introduced in this work, WICUS is an ongoing effort, thus we also plan to extend the ontology network to include new concepts and relations such as software variants, incompatibilities, and user policies for resource consumption.




