\section{Introduction}

Reproducibility of results of published scientific experiments is a cornerstone in 
science. Therefore, the scientific community has been encouraging researchers 
to publish their contributions in a verifiable and understandable 
way~\cite{YaleRoundtable09, James-XSEDE-2014}. In computational science, 
or \emph{in-silico} science, reproducibility often requires that researchers make 
code and data publicly available so that the data can be analyzed in a similar 
manner as in the original work described in the publication. Code must be available 
to be distributed, and data must be accessible in a readable format~\cite{bookReproducibility}. 

In the context of scientific experiments, terms such as reproducibility, replicability, and repeatability are sometimes used as synonymous. Even though there is no clear consensus on how to define both (definitions may vary over different scientific areas), in this work we understand them as separated concepts~\cite{Drummond2011}. In this work, we address the reproducibility of the execution environment for a scientific workflow, as we do not aim to obtain an exact incarnation of the original one, but rather an environment that is able to support the required capabilities exposed by the former environment.

Scientific workflows are a useful representation for managing the execution of large-scale computations. Many scientists now formulate their computational problems as scientific workflows running on distributed computing infrastructures such as campus Clusters, Clouds, and Grids~\cite{workflowBook}. This representation not only facilitates the creation and management of the computation but also builds a foundation upon which results can be validated and shared. Since workflows formally describe the sequence of computational and data management tasks, it is easy to trace the origin of the data produced. Many workflow systems capture provenance at runtime, which provides the lineage of data products and as such underpins the whole of scientific data reuse by providing the basis on which trust and understanding are built. A scientist would be able to look at the workflow and provenance data, retrace the steps, and arrive at the same data products. 
However, this information is not sufficient for achieving full reproducibility.

Currently, most of the approaches in computational science conservation, in particular for scientific workflow executions, have been focused on data, code, and the workflow description, but not on the underlying infrastructure---which is composed of a set of computational resources (e.g. execution nodes, storage devices, and networking) and software components. We identify two approaches for conserving the environment of an experiment: \emph{physical conservation}, where the real object is conserved due to its relevance and the difficulty in obtaining a counterpart; and \emph{logical conservation}, where objects are described in such a way that an equivalent one can be obtained in a future experiment.

The computational environment is often conserved by using the physical approach, where computational resources are made available to scientists over a sustained period of time. As a result, scientists are able to reproduce their experiments in the same environment. However, such infrastructures demand huge maintenance efforts, and there is no guarantee that it will not change or suffer from a natural decay process~\cite{Gavish2011637}. Furthermore, the infrastructure may be subjected to organization policies, which restricts its access to a selective group of scientists, thus limiting reproducibility to this restricted group. On the other hand, data, code, and workflow description can be conserved by using a logical approach that is not subjected to natural decay processes. 

Accordingly, we propose a logical-oriented approach to conserve computational environments, where the capabilities of the resources (virtual machines (VM)) are described. From this description, any scientist, interested in reproducing an experiment, will be able to reconstruct the former infrastructure (or an equivalent one) in any Cloud computing infrastructure (either private or public). One may argue that it would be easier to keep and share VM images with the community research through a common repository, however the high storage demand of VM images remains a challenging problem~\cite{Mao:2014:ROD:2600090.2512348,6552826}. 

Our approach uses semantic-annotated workflow descriptions to generate lightweight scripts for an experiment management API that can reconstruct the required infrastructure. We propose to describe the resources involved in the execution of the experiment, using a set of semantic vocabularies, and use those descriptions to define the infrastructure specification. This specification can then be used to derive the set of instructions that can be executed to obtain a new equivalent infrastructure. We conduct a practical evaluation for a real scientific workflow application in which we describe the application and its environment using a set of semantic models, and use an experiment management tool to reproduce a workflow execution in two different Cloud platforms. 

In this work we entail reproducibility from the execution environment point of view. For  the sake of showing how our approach works, we provide an example in which we have also include the data and the workflow execution of the experiment.

The paper is organized as follows. Section~\ref{sec:semantic} describes our semantic approach for documenting computational infrastructures. Section~\ref{sec:reproducibility} presents the practical evaluation and the description of the tools used to implement the semantic models and manage the experiment. Section~\ref{sec:related-work} presents the related work, and Section~\ref{sec:conclusion} summarizes our results and identifies future works.
