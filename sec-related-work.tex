\section{Related Work}
\label{sec:related-work}

A computational experiment involves several elements that must be conserved to ensure reproducibility. Most of the works addresses the conservation of data and the workflow description, however the computational environment is often neglected. 
An study to evaluate reproducibility in scientific workflows is conducted in~\cite{zhao2012}. The study evaluates a set of domain-specific workflows, available in the myExperiment~\cite{myExperiment} collaborative environment, to identify causes of workflow decays. The study shows that nearly 80\% of the workflows cannot be reproduced, and that about 12\% are due to the lack of information about the execution environment, and that 50\% are due to the use of third-party resources such as web services and databases. Note that some of those third-party resource issues could be also considered as execution environment problems. 

The Executable Paper Grand Challenge~\cite{elsevierchallenge} and the SIGMOD conference in 2011~\cite{SIGMOD} highlighted the importance of allowing the scientific community to reexamine an experiment execution. The conservation of virtual machine (VM) images emerges as a way of preserving the execution environment~\cite{Brammer,SHARE}. However, the high storage demand of VM images remains a challenging problem~\cite{Mao:2014:ROD:2600090.2512348,6552826}. Moreover, the cost of storing and managing data in the Cloud is still high, and the execution of high-interactivity experiments through a network connection to remote virtual machines is also challenging. A list of advantages and challenges of using VMs for achieving reproducibility is exposed in~\cite{Howe2012}. ReproZip~\cite{reprozip} is a provenance-based tool that tracks operating system calls to identify the libraries and data dependencies, as well as the configuration parameters involved in an experiment. The tool combines all these dependencies into a single package that can be used to reproduce an experiment. Although this approach avoids storing VM images, it still requires storing the application binaries and their dependencies. Instead, our work uses semantic annotations to describe these dependencies.

Software components cannot be preserved just by maintaining their binary executable code, but by guaranteeing the performance of their features. In~\cite{Matthews}, the concept of adequacy is introduced to measure how a software component behaves relatively to a certain set of features. Our work is based on this same concept, where we build a conceptual model to semantically annotate the relevant properties of each software component. Then, we use scripting to reconstruct an equivalent computational environment using these annotations.

A recent and relevant contribution to the state of the art of workflow preservation is being developed within the context of the TIMBUS project~\cite{timbus}. The project  aims to preserve and ensure the availability of business processes and their computational infrastructure, aligned with the enterprise risk and the business continuity managements. They also propose a semantic approach for describing the execution environment of a process.  Even though TIMBUS has studied the applicability of their approach to the eScience domain, their approach is mainly focused on business processes.
