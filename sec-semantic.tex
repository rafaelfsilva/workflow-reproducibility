\section{Semantic Modeling of Computational Resources}
\label{sec:semantic}

%\feedback{@Oscar: (following paragraph) this paragraph is actually summarizing and taking for granted many assumptions and preexisting knowledge. Expand it}

%--Deleted base on oscar's feedback
%Scientific workflows are also used for preserving and sharing scientific experiments in science. 


%Some previous research efforts have focused on describing the workflow structure and the 
%experimental data, both input data and results. 

In this work, we argue that in order to achieve reproducibility of a scientific workflow, enough information 
about the computational resources should be provided. These descriptions allow the target audience, 
usually another computational scientist in the same domain, to better understand the underlying 
components involved in a workflow execution.

\rev{\paragraph{\textbf{WICUS ontology networkx}}}

We define semantic models for describing the main domains of a 
computational infrastructure, and for defining the taxonomy of concepts and the relationships 
between them. These models describe software components, hardware specifications, 
and computational resources (in the form of VMs). They also capture infrastructure 
dependencies of the workflows (e.g services that must be running, available libraries, etc.).
 
 As a result, this process facilitates experiment's reusability since 
a new experiment, which may reuse parts of the workflow previously modeled, or a reproduction 
of a workflow, would benefit from the infrastructure dependencies already described.

We have identified four main domains of interest for documenting computational scientific 
infrastructures~\cite{wicus} and developed a set of models, one for each domain, 
and an ontology network that defines the inter-domain relations between these models 
(Figure~\ref{fig:wicusrels}):

\begin{figure}[!t]
	\centering
	\includegraphics[width=.9\linewidth]{figures/wicusrels}
	\caption{Overview of the ontology network ($\rightarrow$ denotes inter-domain relation).}
	\label{fig:wicusrels}
\end{figure}

\begin{itemize}
	\setlength{\itemsep}{1pt}
	\setlength{\parskip}{0pt}
	\setlength{\parsep}{0pt}

	\item{\emph{Hardware domain}}: it identifies the most common hardware information, 
		including CPU, Storage and RAM memory, and their capacities.
	
	\item{\emph{Software domain}}: it defines the software components involved on the execution. 
    		It includes the pieces of executable software (e.g., scripts, binaries, and libraries) used in 
		the experiment. In addition, dependencies between those components and configuration 
		information are also defined, as well as the required steps for deploying them.
	
	\item{\emph{Workflow domain}}: it describes and relates workflow fragments (a.k.a transformations) 
    		to their dependencies. Therefore, scientists can understand which are the relevant infrastructure 
		components for each part of the workflow.
	
	\item{\emph{Computing Resources domain}}: it expresses the information about the available 
    		computing resources. In this domain, only virtualized resources are currently considered 
		(i.e., virtual machine). It includes the description of the VM image, its provider, and specifications.
\end{itemize}


\rev{The Workflow Infrastructure Conservation Using Semantics ontology 
(WICUS) is an OWL2 (Web Ontology Language) ontology network that 
implements the conceptualization of these domains. This ontology 
network is available online\footnote{http://purl.org/net/wicus} and its goal is to define the relevant and 
required properties for describing scientific computational infrastructures. 
The detailed description of the ontologies, including their main terms and relation
in the context of a workflow execution are provided in~\cite{wicus}.
Currently, two versions of the ontology network have been released. The latest one, released in
August 2014,  includes a set of new properties for better describing software and hardware requirements, 
and also for including the output information of a configuration process (e.g., the resultant IP and port on
which a recently deployed service will be listening).}


\rev{These models have been documented and published online, and have been also aligned with several of well-known vocabularies, such as p-plan\footnote{http://purl.org/net/p-plan} and DCMI Metadata Terms\footnote{http://dublincore.org/documents/dcmi-terms/}. For example, the class wreqs:Workflow that represents a scientific workflow in our domain, has been aligned as a subclass of p-plan:Plan, which in turn is a subclass of prov:Plan, from the PROV vocabulary\footnote{http://www.w3.org/ns/prov}. Further versions of the WICUS ontology network might be aligned with new vocabularies}.  

\rev{Listing \ref{lst:wicus-sample} illustrates a running example, summarizing the main annotations of the SoyKB workflow requirements.  From line 1 to 5 states the individual \path{Workflow:soykb_WF} as a member of the \path{wreqs:Workflow} class and indicates that it requires the \path{SoftwareRequirements:soykb_WF_SOFT_REQ} requirement as part of it execution environment. Lines 6-11 define the different parts of the main workflow which will define their own requirements. From lines 13 to 16, we define the workflow software requirement to be composed by the \path{SoftwareStack:PEGASUS_WMS_CENTOS_6_5_SOFT_STACK} software stack, and that this stack depends on other several software stack, such as Condor, SSH or the Java SDK (lines 24-29).}



%%add examples ttl here

\begin{lstlisting}[caption={WICUS workflow annotations example.}, label={lst:wicus-sample}]
Workflow:soykb_WF
 a wreqs:Workflow ;
  wreqs:requiresExecutionEnvironment
   HardwareRequirements:soykb_WF_HW_REQ,
   SoftwareRequirements:soykb_WF_SOFT_REQ.
 wreqs:hasSubworkflow
  ConcreteWorkflow:SELECT_VARIANTS_INDEL_CONC_WF, 
  ConcreteWorkflow:SORT_SAM_CONC_WF, 
  ConcreteWorkflow:INDEL_REALIGN_CONC_WF, 
   ...
  ConcreteWorkflow:FILTERING_SNP_CONC_WF,      
   
SoftwareRequirements:soykb_WF_SOFT_REQ
 a wreqs:SoftwareRequirements ;
 wicus:composedBySoftwareStack
  SoftwareStack:PEGASUS_WMS_CENTOS_6_5_SOFT_STACK.
  

SoftwareStack:PEGASUS_WMS_CENTOS_6_5_SOFT_STACK
 a wstack:SoftwareStack ;
 wstack:dependsOn
  SoftwareStack:CENTOS_6_5_OS_SOFT_STACK 
   ...
  SoftwareStack:JAVA-1.7.0-OPENJDK.X86_64_SOFT_STACK 
  SoftwareStack:CONDOR_CENTOS_6_5_SOFT_STACK 
  SoftwareStack:OPEN_SSH_SERVER_SOFT_STACK 
  SoftwareStack:OPEN_SSH_CLIENTS_SOFT_STACK ;
 wstack:hasSoftwareComponent
  SoftwareComponent:PEGASUS_WMS_CENTOS_6_5_SOFT_COMP .


\end{lstlisting}



\rev{\paragraph{\textbf{Abstract Deployment Plan}}}
\rev{This layer allows WICUS to generate abstract deployment plans regardless of the
underlying execution tool. The abstract plan is based on the WICUS Software~\cite{wicus} 
domain ontology, which defines the software stacks that should be deployed in the 
execution platform. Figure~\ref{fig:stack-rel} shows the relationships between the different
elements that compose the \texttt{Stack} domain ontology. A \texttt{Software Stack} may
be composed of one or more \texttt{Software Components}. Each of them has an associated 
\texttt{Deployment Plan} according to the target execution platform, which is composed of 
one or more \texttt{Deployment Steps}.

\begin{figure}[!htb]
	\centering
	\includegraphics[width=0.9\linewidth]{figures/stack-rel}
	\caption{Overview of the WICUS software stack relation diagram.}
	\label{fig:stack-rel}
\end{figure}

Listing \ref{lst:plan-soykb} shows an example of the abstract plan for the SoyKB workflow generated by the ISA. 
The first section of the plan (lines 1--26) describes the deployment of the \texttt{Pegasus 
WMS} and its related dependencies. Note that this section is common across all deployment
plans for the workflows covered in this work. The remaining lines describe how the SoyKB 
software is deployed. The \texttt{SOFTWARE.TAR.GZ} stack, which is a dependency for all 
SoyKB wrappers, is the first component to be deployed (lines 27--29). Finally, the last section 
of the plan (lines 30--45) describes how the four SoyKB wrappers are deployed. For each
wrapper, two deployment steps are required: 1)~copy of the program execution binary, and
2)~the granting of proper execution permissions.}

\begin{lstlisting}[caption={Abstract deployment plan of the SoyKB WF.}, label={lst:plan-soykb}]
OPEN_SSH_CLIENTS_SOFT_STACK stack
  OPEN_SSH_CLIENTS_SOFT_COMP component
    OPEN_SSH_CLIENTS_DEP_STEP step
OPEN_SSH_SERVER_SOFT_STACK stack
  OPEN_SSH_SERVER_SOFT_COMP component
    OPEN_SSH_SERVER_DEP_STEP step
WGET_SOFT_STACK stack
  WGET_SOFT_COMP component
    WGET_DEP_STEP step
CONDOR_CENTOS_6_5_SOFT_STACK stack
  CONDOR_CENTOS_6_5_SOFT_COMP component
    STOP_CONDOR_DEP_STEP step
    ADD_CONDOR_REPO_DEP_STEP step
    CONDOR_YUM_INSTALL_DEP_STEP step
    CLEAN_AND_SET_CONDOR_DEP_STEP step
    RESTART_DAEMONS_DEP_STEP step
JAVA-1.7.0-OPENJDK.X86_64_SOFT_STACK stack
  JAVA-1.7.0-OPENJDK.X86_64_SOFT_COMP component
    JAVA-1.7.0-OPENJDK.X86_64_DEP_STEP step
JAVA-1.7.0-OPENJDK-DEVEL.X86_64_SOFT_STACK stack
  JAVA-1.7.0-OPENJDK-DEVEL.X86_64_SOFT_COMP component
    JAVA-1.7.0-OPENJDK-DEVEL.X86_64_DEP_STEP step
PEGASUS_WMS_CENTOS_6_5_SOFT_STACK stack
  PEGASUS_WMS_CENTOS_6_5_SOFT_COMP component
    ADD_PEGASUS_REPO_DEP_STEP step
    PEGASUS_YUM_INSTALL_DEP_STEP step
SOFTWARE_TAR_GZ_SOFT_STACK stack
  SOFTWARE_TAR_GZ_SOFT_COMP component
    SOFTWARE_TAR_GZ_DEP_STEP step
PICARD-WRAPPER_SOFT_STACK stack
  PICARD-WRAPPER_SOFT_COMP component
    PICARD-WRAPPER_DEP_STEP step
    PICARD-WRAPPER_2_DEP_STEP step
SOFTWARE-WRAPPER_SOFT_STACK stack
  SOFTWARE-WRAPPER_SOFT_COMP component
    SOFTWARE-WRAPPER_DEP_STEP step
    SOFTWARE-WRAPPER_2_DEP_STEP step
GATK-WRAPPER_SOFT_STACK stack
  GATK-WRAPPER_SOFT_COMP component
    GATK-WRAPPER_DEP_STEP step
    GATK-WRAPPER_2_DEP_STEP step
BWA-WRAPPER_SOFT_STACK stack
  BWA-WRAPPER_SOFT_COMP component
    BWA-WRAPPER_DEP_STEP step
    BWA-WRAPPER_2_DEP_STEP step
\end{lstlisting}



